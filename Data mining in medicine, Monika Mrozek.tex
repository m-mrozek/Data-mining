\documentclass{beamer}
\usepackage[utf8]{inputenc}
\usepackage[T1]{fontenc}
\usepackage{listings}
\usepackage{graphics}
\usepackage{amsmath}
\lstset{
   showstringspaces=false
  basicstyle=\ttfamily,
  columns=fullflexible,
  frame=false,
  breaklines=true,
}


\usetheme{Warsaw}

\begin{document}
\title{Optimization in object classification}
\subtitle{The multisurface method}
\author{Monika Mrozek} 
\date{11 December 2019} 

\frame{\titlepage} 


\section{Introduction}

\subsection{Xcyt: the system for remote cytological diagnosis}
\begin{frame}{Xcyt}{Basic assumptions and functions}
The main aim of the software named \textbf{Xcyt} was expanding the possibilities of the prognosis and diagnosis of breast cancer. The system was invented at the University of Wisconsin in Madison\cite{s}. It combines many techniques: digital image analysis, inductive machine learning, mathematical programming and statistics.

\begin{block}{Assumptions and functions}
\begin{itemize}
   \item analysis of cytological features,
   \item diagnosis and attempt to estimate the probability of malignancy,
   \item when the sample is cancerous, prediction of recurrence.
\end{itemize}
\end{block}
\end{frame}


\begin{frame}{Xcyt}{Cellular features}
\textbf{Xcyt} makes possible to analyse the exact boundaries of the nuclei. It uses curve-fitting program 
and examines a lot of parameters:
\begin{block}

\begin{itemize}
   \item area,% -- jest to liczba pikseli z których składa się obraz jądra powiększona o $\frac{1}{2}$ liczby pikseli składających się na jego otoczkę,
   \item radius,
   \item perimeter,
   \item symmetry,
   \item number and size of concavities,
   \item fractal dimension,
   \item compactness,
   \item smoothness.
   
\end{itemize}
\end{block}


\end{frame}



\section{Multisurface Method-Tree (MSM-T)}
\subsection{Principal assumptions}
\begin{frame}[allowframebreaks]{Multisurface Method-Tree}{Building of a model}

Xcyt is closely connected with \textbf{MSM-Tree} (\textit{Multisurface Method-Tree}). Its main idea is separation of the points from matrices \textbf{A} and~\textbf{B}. In these circumstances

\begin{itemize}
\item \textbf{A$_{m \times n}$} is a matrix containing $m$ vectors $x \in \mathbb{R}^n$ with data about patients who were diagnosed with malignant tumour,
\item \textbf{B$_{k \times n}$} is a matrix containing $k$ vectors $x \in \mathbb{R}^n$ with data about patients who were diagnosed with benign tumour.
\end{itemize}



The separating plane is represented by \cite{bcd}
\begin{align*}
\gamma=x^T \cdot w \in \mathbb{R}^n
\end{align*}
if and only with \\
$A w \geq e \gamma + e$ \\
$B w \leq e \gamma - e$\\
where 
\begin{itemize}
\item $w$ - normal to separating plane, $w \in \mathbb{R}^n$
\item e - vector of ones of appropiate dimension.
\end{itemize}



\end{frame}



\begin{frame}{Multisufrace Method-Tree}{Realtionship with linear programming}
The problem of separating group of points by piecewise-linear surface is equivalent to solving a linear program
\begin{align*}
\displaystyle \min_{w,\gamma,y,z}\frac{e^Ty}{m}+\frac{e^Tz}{k}
\end{align*}
subject to
\begin{itemize}
\item $A w \geq e \gamma + e$ 
\item $B w \leq e \gamma - e$
\item $y,z \geq 0$
\end{itemize}
\end{frame}





\subsection{Visualisation of MSM-T}
\begin{frame}
\begin{figure}[h]
	\caption{\textit{Olvi L. Mangasarian, W. Nick Street, William H. Wolberg: Breast Cancer Diagnosis via Linear Programming}, Mathematical Programming Technical \cite{bcd} }
	\centering
	\includegraphics[scale=0.4]{K}
	\caption{Separating planes in MSM-T}\label{rys33}
	
\end{figure}
\end{frame}


\section{Clinical usage -- diagnosis of breast cancer}
\subsection{Diagnostic problem}
%\begin{frame}{Application of MSM-T in breast cancer diagnosis}
\textbf{MSM} method has a signficant participation in improving breast cancer treatment. The scientists used the diagnosis
of breast cytology to demonstrate the applicability of this
method to medical diagnosis and decision making. \\ 


\begin{figure}[h]

	\caption{\textit{William H. Wolberg, Olvi L. Mangasarian: Multisurface method of pattern separation for medical diagnosis applied to breast cytology }\cite{msm} }
	\centering
	\includegraphics[scale=0.4]{H}
	\caption{FNA cytology}
	
\end{figure}




\subsection{Procedure}
\begin{frame}[allowframebreaks]{Description of procedure}
The experiment was conducted at the University of Wisconsin Hospitals. The scientists analised 370 samples. Complete classification was obtained after 4 steps.
\begin{block}

\begin{itemize}
\item $(x_1,x_2, \ldots, x_9) \in \mathbb{R}^9$ - the vector of real numbers created for each sample, which contains nine cytological characteristics,
\item  $(c_{i1},c_{i2}, \ldots, c_{i9}) \in \mathbb{R}^9$ -the normal to the plane, $i=1,2,3,4$
\item $a_i,b_i$- the distance from the origin to the planes  $\in \mathbb{R}^9$
\end{itemize}

\end{block}
Correct separation was accomplished in 369 of 370 samples (201 benign
and 169 malignant).
\begin{block}{Interpretation of results}

For $i=1,2,3$ we have
\begin{itemize}
\item $c_{i1}x_1+c_{i2}x_2+\ldots+c_{i9}x_9>b_i$ - $i$, then  tumour is malignant at stage $i$,
\item $c_{i1}x_1+c_{i2}x_2+\ldots+c_{i9}x_9<a_1$ - $i$, then  tumour is benign at
stage $i$.
\end{itemize}

Additionally

\begin{itemize}
\item $c_{41}x_1+c_{42}x_2+\ldots+c_{49}x_9\geq(a_4+b_4)/2$ - then tumour is malignant at stage 4,
\item $c_{41}x_1+c_{42}x_2+\ldots+c_{49}x_9<(a_4+b_4)/2$ - then tumour is benign at stage 4.
\end{itemize}


\end{block}



\end{frame}









%\section{Informacja dotycząca oprogramowania}


%\begin{frame}[fragile,allowframebreaks]{Oprogramowanie}{Dane}
%W najbardziej popularnych programach statystycznych można znaleźć zbiory danych powstałe dzięki metodzie \textbf{MSM} (MSM-T). 

%\begin{block}{Breast Cancer Wisconsin (Diagnostic) Data Set}
%Dane zebrane przez dr. Williama H. Wolberga: 
%\textit{https://archive.ics.uci.edu/ml/machine-learning-databases/breast-cancer-wisconsin/} \\ 
%\vspace{0.3cm}

%Informacje o danych \\
%\textit{https://archive.ics.uci.edu/ml/datasets/
%Breast+Cancer+Wisconsin+(Diagnostic)}


%\end{block}

%\begin{block}{RStudio}
%Dane dot. wyników biopsji dla 699 pacjentów
%\begin{lstlisting}[language=R]
%library(MASS)
%data(biopsy)
%\end{lstlisting}

%\end{block}

%\begin{block}{MATLAB}
%Generowanie drzewa decyzyjnego
%\begin{lstlisting}[language=Matlab]
%T = msmt_tree(A,B,max_depth,tolerance,
%certainty_factor,min_points) 

%\end{lstlisting}



%\end{block}

%\end{frame}





\begin{frame}{Example of decision tree}

\begin{figure}[h]

	\caption{\textit{Multisurface Method Tree with MATLAB \url{http://pages.cs.wisc.edu/~olvi/uwmp/msmt.html}}}
%	\centering
	\includegraphics[scale=0.4]{T}
	\caption{Graphical ilustration of the decision tree for Wisconsin Breast Cancer data}
	
\end{figure}



\end{frame}














































































\begin{frame}
\frametitle{Bibliography}
\begin{thebibliography}{9}
\bibitem{bcd}
Olvi L. Mangasarian, W. Nick Street, William H. Wolberg: \emph{Breast Cancer Diagnosis via Linear Programming}, Mathematical Programming Technical Report 94-10, Decemeber 19, 1994

\bibitem{s}
W. N. Street: \emph{Xcyt: A System for Remote Cytological Diagnosis and Prognosis of Breast Cancer},
Management Sciences Department, University of Iowa, Iowa City, IA USA

\bibitem{msm}
William H. Wolberg, Olvi L. Mangasarian: \emph{Multisurface method of pattern separation for medical diagnosis applied to breast cytology}, Proc. Natl. Acad. Sci. USA, December 1990 \url{https://pdfs.semanticscholar.org/5053/7a16eb18e2bc4971165258cba7a071f38cb7.pdf}

\end{thebibliography}
\end{frame}



















\end{document}
